\documentclass[a4paper]{article}

%% Language and font encodings
\usepackage[english]{babel}
\usepackage[utf8x]{inputenc}
\usepackage[T1]{fontenc}

%% Sets page size and margins
\usepackage[a4paper,top=3cm,bottom=2cm,left=2.5cm,right=2.5cm,marginparwidth=1.75cm]{geometry}

%% Useful packages
\usepackage{amsmath}
\usepackage{graphicx}
\usepackage[colorinlistoftodos]{todonotes}
\usepackage[colorlinks=true, allcolors=blue]{hyperref}

\title{Rush Hour Algoritmen en Heuristieken}

\author{
  Martijn, de Jong\\
  \textit{martijn.dejong@mail.com}\\
  \textit{Universiteit van Amsterdam}
  \and
  Thomas, Kamps\\
  \textit{thomas@kamps.email}\\
  \textit{Universiteit van Amsterdam}
  \and
  Rick, Vergunst\\
  \textit{raikos1337@gmail.com}\\
  \textit{Universiteit van Amsterdam}
}

\begin{document}
\maketitle

\begin{abstract}
Your abstract.

\vspace{5mm}
\textbf{Keywords:} Rush Hour, Heuristieken, Depth-First Search, Breadth-First Search, Best-First Search, A*, Beam Search
\end{abstract}

\section{Introduction}
\textit{relevantie, interesseren, aanleiding, scope, referenties}

Dit onderzoeksverslag in een beschrijving van het proces dat doorlopen is tijdens vak Heuristieken aan de Universiteit van Amsterdam. Het doel van dit onderzoek is het leren benaderen, beschrijven, categorieseren en daardoor uiteindelijk oplossen van een complex probleem. Het vakgebied heursitieken (Grieks: heuriskein = vinden) is de wetenschap die zich bezig houdt met het vinden van oplsossingen voor deze zeer complexe problemen. In de praktijk is dit bijvoorbeeld terug te zien in de computerbeveiliging, waarbij nu nog onoplosbare problemen gebruikt worden om bijvoorbeeld betaalverkeer te versleutelen.

\section{Rush Hour}
Het spel Rush Hour is een schuifpuzzel dat ontwikkeld is door Nobuyuki Yoshigahara in de jaren zeventig en op de markt gebracht door Thinkfun. Het speelbord heeft een raster van 6 bij 6 waarop auto's en vrachtwagens geplaatst zijn met respectievelijk een lengte van 2 en 3. Deze voertuigen kunnen zowel horizontaal als verticaal op het raster geplaatst worden. Op de derde rij vindt zich aan de rechter kant een uitgang waar een auto het speelbord kan verlaten. Elk begin bord bevat één rode auto horizontaal geplaatst op de derde rij. Daarnaast zijn er een variabel aantal auto's en vrachtwagens op dit raster geplaatst in beide orientaties. Elk voertuig kan zowel vooruit als achteruit bewegen in de orientatie waarin dit voertuig staat. Digagonaal verplaatsen is dus niet toegestaan. Verplaatsing van een voertuig kan ook niet als het voertuig gehinderd wordt door een ander voertuig. Het doel is om de rode auto door de uitgang uit het speelbord te laten rijden. 

\subsection{Probleem beschrijving}
\textit{Beschrijving probleem. probleemsets, problemspace, constraints satisfaction problem}
Dit probleem is door Flake en Baum \cite{flake2002rush} beschreven als een PSPACE-complete probleem. Dit betkent...

De grootte van de toestandsruimte kan erg verschillen per probleeminstantie. Voor de configuraties die gebruikt worden in het onderzoek van Jarušek en Pelánek \cite{jaruvsek2011determines} kan dit varieeren tusssen de 600 en 80.000 staten. Dit betreffen echter alleen 6x6 spelborden. Voor de grotere borden is deze ruimte groter, OMDAT . De toestandsruimte is altijd ongericht, doordat alle bewegingen die gemaakt worden weer ongedaan kunnen worden gemaakt.

Een ondergrens is voor elk spelbord duidelijk aan te wijzen, dat is namelijk het aantal 

\subsection{Opdracht}
Deze opdracht zoals gegeven bij het vak Heuristieken bestaat uit verschillende delen. Opdracht 1 bestaat uit het oplossen van drie gegeven 6 bij 6 speelborden. Dit mag door middel van een script gebeuren, maar ook met de hand. De tweede opdracht bestaat uit drie speelborden van 9 bij 9. Deze borden moeten worden opgelost doormiddel van een eigen geschreven script. Hierbij geldt dat een oplossing met minder stappen een betere oplossing is. Opdracht 3 bestaat uit één bord van 12 bij 12 waarbij het draait om het optimaliseren van je script van oprdacht twee. Deze beginborden zijn te vinden in de \textbf{apendix}. Tot slot is er nog een extra opdracht waarbij gekeken moet worden naar de factoren die de moeilijkheidsgraad van een spelbord bepalen. Hierbij kunnen verschillende spelborden met dezelfde afmeting vergeleken worden.

\section{Methoden}
Voor het oplossen van de spelborden is gekeken naar zowel diverse bestaande algoritmen als zelfbedachte proceduren. Er is begonnen met de simpele regel dat de rode auto altijd 1 stap naar voren beweegt. Indien deze gehinderd wordt, beweegt de auto waardoor deze hinderd wordt 1 stap. Dit gaat door totdat de rode auto een stap naar voren kan bewegen. Deze procedure heeft echter een aantal beperkingen zoals een loop waarin het script tercht kan komen waarbij meerdere auto's continu in dezelfde volgorde dezelfde stap zetten. Dit is eventueel op te lossen met het bijhouden van een archief. Daarnaast is het in de praktijk soms ook nodig om de rode auto naar achter te bewegen of te laten staan om andere auto's te laten passeren. Ook is gekeken naar voorrang voor vrachtwagens op de auto's bij bewgen, maar hierbij hetzelfde probleem. Vervolgens zijn diverse bestaande algoritmen onderzocht en uitgewerkt.

\subsection{Depth-First Search}
Het depth-first search (DFS) algoritme werkt langs een boomstructuur en werkt de takken één voor één uit. Vanuit het begin spelbord, de beginstaat, worden voor elke mogelijke beweging van elk voertuig een apart spelbord gecreerd, een nieuwe staat. Deze spelborden vormen de kinderen van het beginspelbord. Deze spelborden worden op een stapel geplaatst, waarbij vervolgens ditzelfde proces herhaalt voor het eerste element van de stapel. Elke staat wordt bijgehouden in een archief, zodat lager in de boomstructuur nooit een staat voorkomt die al eerder bereikt is (dit kan wel bij een graafstructuur). Dit is noodzakelijk zodat er geen loops kunnen ontstaan en omdat de staat al eerder bereikt is met een korter pad, waardoor een mogelijk gevonden oplossing een onnodig lang pad heeft. Indien een bereikte staat geen  kinderen heeft die niet eerder bereikt zijn, wordt er een stap terug gedaan in de boomstructuur en gaat het proces verder voor het tweede element van de stapel op dat niveau in de boomstructuur.

	Dit algoritme is in twee varianten toegepast, zowel iteratief als recursief. Bij de iteratieve aanpak wordt een gedeelte van de code herhaalt door middel van een 'while loop' totdat aan de voorwaarde wordt voldaan. Bij de recursieve aanpak roept de functie zichzelf aan indien niet aan de voorwaarde voor een oplossing wordt voldaan. 
    
    Een voordeel van dit brute-force algoritme is dat het compleet is en dus voor elke input een oplossing terug geeft of aangeeft dat er geen oplossing is. Door het gebruik van een archief kan er ook backtracking plaatsvinden waardoor de stap terug gedaan kan worden en indien een oplossing gevonden is het pad naar de oplossing te herleiden is.  Een nadeel van een DFS-algoritme is echter dat de eerste oplossing die gevonden wordt, niet altijd de korste oplossing is (de oplossing met het minst aantal verschuivingen van voertuigen op het spelbord). Daarnaast komt het voor dat dit algoritme een tak geheel uitwerkt en herhaaldelijk op een 'dood eind' stuit waarbij geen kinderen meer zijn. Hierdoor kan het algoritme aanzienlijk veel tijd in beslag nemen om tot een eerste oplossing te komen. Door middel van 'pruning' kunnen paden die niet tot een oplossing leiden afgekapt worden. Bij dit probleem is echter niet te bepalen hoe goed of slecht een staat is en daarbij de kans dat het gevolgde pad tot een oplossing leidt. Wij hebben dan ook gekozen om dit niet toe te passen.
	
\subsection{Breadth-First Search}
Het Breadth-First Search (BFS) algoritme werkt ook langs een boomstructuur, net als het DFS-algoritme. In plaats van de takken stuk voor stuk uit te werken, werkt het BFS-algoritme in lagen de boomstructuur af. Vanuit het begin bord worden alle kinder staten berekend en in een rij geplaatst. Vervolgens wordt voor elk element in de rij weer de kinder staten berekend. Hierbij wordt dus het first-in-first-out principe toegepast. De 'boom' van dit algoritme groeit dus zeer snel in de breedte en niet zo snel in de diepte als bij het DFS-algoritme. 
	
    Het grootste voordeel van dit algorimte is dat de eerst gevonden oplossing altijd een oplossing is met het korste pad is. Het is mogelijk dat er meer oplossingen zijn met een even lang pad, maar nooit korter. Een nadeel kan echter zijn dat dit algoritme zeer intensief is doordat met elke laag van de boom het aantal mogelijke staten zeer sterk toeneemt. Doordat voor elke staat de mogelijke kinder staten berekend worden, is dit aanzienlijk meer dan bij een DFS algoritme.

\subsection{Best-First Search, A* en Beam Search}
Best-First Search is een variatie op het DFS algoritme waarbij heuristieken worden toegepast. Heuristieken zijn regels waarvan men vooraf niet weet of dit een verbetering zal opleveren van het algoritme bij het oplossen van een specifiek probleem. Voor elke staat wordt een prijs berekend op basis van deze heuristieken. Zo krijgt bijvoorbeeld een spelbord waarbij er drie voertuigen de weg voor de rode auto blokkeren een hogere prijs dan een spelbord waar maar één voertuig de weg blokeert. Vervolgens zullen de kinder staten van de staat met de laagste prijs berekend worden. In de volgende sectie zal er uitgebreider ingegaan worden op de verschillende heuristieken die toegepast zijn. A* is ook een variatie op een eerder genoemd algoritme, namelijk het BFS. Ook hierbij worden heuristieken toegepast.
	
    Beam Search is een combinatie van zowel DFS, BFS en A*. Hierbij wordt in plaats van alleen de staat met de laagste prijs, van een paar staten met de laagste prijs de kinderen berekend. Van deze kinder staten worden vervolgens weer de staten met de laagste prijs verder uitgewerkt. De boom bij dit algoritme zal altijd een vooraf bepaalde breedte behouden. Ten opzichten van een BFS algoritme kan het geheugengebruik een stuk lager zijn doordat er een beperkt aantal borden bekeken wordt en dus in het archief terecht komen.

\subsection{Heuristieken}
Op verschillende eerder besproken algoritmen zijn diverse heuristieken toegepast. Een heuristiek garandeert niet altijd een optimale oplossing, maar kan wel zorgen voor een aanzienlijke afname van het aantal bekeken staten en daarmee de tijd waarin het spelbord opgelost wordt. De heuristieken die wij toegepast hebben verdelen we dan ook in twee categorieen: admissible en inadmissible. Bij het toepassen van een admissible heuristiek zal de garantie van het vinden van het pad met de laagste kosten blijven indien het algoritme dit ook grandeert. Een inadmissible heursitiek heeft deze garantie echter niet. Desondanks is er toch ook gekozen om deze heuristieken toe te passen, omdat deze bij een groter spelbord ervoor kan zorgen er in ieder geval binnen een redelijke tijd een oplossing gevonden kan worden.
	
    Het eerste heuristiek dat is toegepast, is de \textit{afstand van de rode auto tot de uitgang}. Hierbij wordt het aantal vakjes geteld tussen de voorkant van de rode auto en het meest rechter vakje op het bord. Om dit te berekenen is gebruik gemaakt van de begin x-coördinaat van de rode auto, de lengte van de auto en de grootte van de rij waarin de auto zich bevindt. Een spelbord waarbij de auto verder van de uitgang staat krijgt een hogere prijs. Afstand van één vakje krijgt een prijs van één, afstand van twee vakjes krijgt een prijs van twee enzovoort. Hierdoor zal met een A* algoritme of Beam Search de staat waarbij de rode auto dichter bij de uitgang is eerder uitgewerkt worden. 

	Dit eerste heuristiek is gebruikt in combinatie met een ander heursitiek, \textit{aantal blokkerende voertuigen}. Dit heuristiek geeft een hogere prijs aan een spelbord indien er meerdere voertuigen de weg naar de uitgang blokkeren voor de rode auto. Een gegeven is dat er tussen de rode auto en de uitgang alleen verticaal voertuigen de weg kunnen blokkeren en nooit horizontaal. In de rij waarin de rode auto zich bevindt wordt gekeken wat de x-coördinaat van de auto is, de lengte wordt vervolgens daarbij opgeteld. Tot slot wordt gekeken of er voertuigen voor de rode auto staan die een hoger x-coördinaat hebben en hoeveel voertuigen. Voor één blokkerend voertuig wordt de prijs één, voor twee blokkerende voertuigen wordt de prijs twee, enzovoort. Als uitbreiding op dit heuristiek is nog bekeken of het \textit{blokkerende voertuig een vrachtwagen of auto} betreft. Door de blokkerende vrachtwagen een lagere prijs te geven dan een blokkerende auto, wordt het bord met een blokkerende vrachtwagen eerder uitgewerkt. 
    
\subsection{Kostenberekening}




\section{Uitwerking}
Voor de uitwerking van de algortimen met toegepasten heuristieken is gekozen om te werken in de programmeertaal Python. Deze keuze is gemaakt omdat alle teamleden al vanuit de opleiding kennis hebben van deze taal en meerdere malen hiermee gewerkt hebben. Daarnaast biedt Python genoeg mogelijkheden voor het bijhouden van queues en stacks, maar ook voor het visualiseren van de spelborden.

\subsection{Grid Class}


\subsection{Archief}


\subsection{Visualisatie}
Voor de visualisatie is gebruik gemaakt van twee bibliotheken: Python Graphics en Pygame. De eerste versie van de visualisatie is gemaakt met behulp van Python Graphics. Hierbij werd het pad van beginbord naar de oplossing dat gevonden was door het algoritme ingeladen. Bij elk opvolgend spelbord in deze reeks zijn steeds twee vakjes anders, namelijk het vakje voor het bewegende voertuig en na dit voertuig. Deze vakjes werden dan opnieuw 'getekend' in het scherm. Een groot nadeel van deze bibliotheek was echter de snelheid. Het bewegende voertuig werd in de visualisatie eerst één vakje langer, en vervolgens werd het vakje aan de achterkant eraf gehaald. 
	De uiteindelijke visualitatie is gemaakt met behulp van Pygame. Deze module kent veel meer opties en scoort op snelheid een stuk beter. Het spelbord wordt wederom op dezelfde manier ingeladen, ditmaal vanuit een extern bestand. In het controle paneel kan gekozen worden welke configuratie de gebruiker wilt zien en welk algoritme toegepast moet worden. De gebruiker kan vervolgens op de 'play' knop drukken om alle stappen naar de oplossing te zien, of er kan gekozen worden om zelf door de stappen heen te klikken. Daarnaast kan het bord gereset worden en kunnen de gegenereerde kleuren aangepast worden. De gebruikte data wordt allemaal opgehaald uit een extern bestand. Daarnaast is de mogelijkheid om een livemodus te gebruiken waarbij de oplossing real-time berekend wordt met het aangegeven algoritme. Tot slot kan in dit controlepanel nog extra informatie afgelezen worden als het aantal gezette stappen en totaal aantal, grootte van het spelbord, aantal auto's en vrachtwagen, lege ruimte op het bord en aantal gecontroleerde spelborden door het algoritme.
    
\section{Resultaten}
Voor het vergelijken van de verschillende algoritmen en heuristieken onderling is gekozen voor het vergelijken op het aantal bekeken spelborden door het algoritme en het aantal stappen dat het algoritme nodig heeft om van het begin spelbord bij het eindbord te komen. Voor deze maat is gekozen omdat dit voor elk algoritme de efficiëntie aangeeft: minder bekeken staten betekent dat het algoritme efficiënter door de oplossingsruimte is bewogen van het beginspelbord naar de oplossing. Ook betekend minder bekeken staten dat er minder spelborden in het archief opgeslagen hoeven te worden en er dus minder snel spraken zal zijn van een 'memory overload'. De tweede maat, het aantal stappen dat nodig is om van het beginbord tot het eindbord te komen, geeft aan of het algoritme een van de optimale oplossing geeft of een pad heeft gevonden dat eigenlijk langer dan nodig is. De tijd die nodig is voor het berekenen van een oplossing is ook bekeken als maat, maar dit kan verschillen per computer en zelfs op dezelfde computer bij meerdere keren uitvoeren. Ondanks dat de tijd wel een belangrijke rol speelt bij het oplossen van de grote borden (9x9 en 12x12), is dit niet gemeten.

	In de onderstaande tabel kunt u het aantal bekeken staten zien per algoritme en per bord.

\section{Analyse}

\section{Discussie}

  

\bibliographystyle{alpha}
\bibliography{sample}

\end{document}